%%%%%%%%%%%%%%%%%%%%%%%%%%%%%%%%%%%%%%%%%
% Article Notes
% Structure Specification File
% Version 1.0 (1/10/15)
%
% This file has been downloaded from:
% http://www.LaTeXTemplates.com
%
% Authors:
% Vel (vel@latextemplates.com)
% Christopher Eliot (christopher.eliot@hofstra.edu)
% Anthony Dardis (anthony.dardis@hofstra.edu)
%
% License:
% CC BY-NC-SA 3.0 (http://creativecommons.org/licenses/by-nc-sa/3.0/)
%
%%%%%%%%%%%%%%%%%%%%%%%%%%%%%%%%%%%%%%%%%

%----------------------------------------------------------------------------------------
%	REQUIRED PACKAGES
%----------------------------------------------------------------------------------------

\usepackage[includeheadfoot,columnsep=2cm, left=1in, right=1in, top=.5in, bottom=.5in]{geometry} % Margins

%\usepackage[T1]{fontenc} % For international characters
\usepackage[utf8]{inputenc}
%\usepackage{XCharter} % XCharter as the main font

\usepackage{natbib} % Use natbib to manage the reference
%\usepackage{cite}
%\bibliographystyle{apalike} % Citation style
%\bibliographystyle{te} % Citation style

\usepackage[english]{babel} % Use english by default
\usepackage{amsmath}
\usepackage{amssymb}
\usepackage{amsthm}
\usepackage{xcolor}
\usepackage{graphicx}
\usepackage{enumerate}
\usepackage{todonotes}
\usepackage{listings}
\usepackage{pythonhighlight}

\newtheorem{df}{Definition}
\newtheorem{ex}{Example}
\newtheorem{as}{Assumption}
\newtheorem{rem}{Remark}
\newtheorem{pr}{Proposition}
\newtheorem{qu}{Question}
\newtheorem{lm}{Lemma}
\newtheorem{thm}{Theorem}

%----------------------------------------------------------------------------------------
%	CUSTOM COMMANDS
%----------------------------------------------------------------------------------------

\newcommand{\articletitle}[1]{\renewcommand{\articletitle}{#1}} % Define a command for storing the article title
%\newcommand{\articlecitation}[1]{\renewcommand{\articlecitation}{#1}} % Define a command for storing the article citation
\newcommand{\doctitle}{``\articletitle''} % Define a command to store the article information as it will appear in the title and header

\newcommand{\datenotesstarted}[1]{\renewcommand{\datenotesstarted}{#1}} % Define a command to store the date when notes were first made
\newcommand{\docdate}[1]{\renewcommand{\docdate}{#1}} % Define a command to store the date line in the title

\newcommand{\docauthor}[1]{\renewcommand{\docauthor}{#1}} % Define a command for storing the article notes author

% Define a command for the structure of the document title
\newcommand{\printtitle}{
\begin{center}
\textbf{\Large{\doctitle}}
\docdate

\docauthor
\end{center}
}
% yout
\newcommand{\yo}{{y_{out}}}
% Gamma out
\newcommand{\Go}{\Gamma_{out}}
% Gamma c
\newcommand{\Gc}{\Gamma_{c}}
% gamma out
\newcommand{\go}{\gamma_{out}}
% gamma c
\newcommand{\gc}{\gamma_{c}}
% L2 scalar product
\newcommand{\ltwosp}[2]{\langle{#1},{#2}\rangle_{L^2(\Omega)}}
% L2 scalar product (explicit)
\newcommand{\ltwospe}[2]{\int_{\Omega} {#1}{#2} \; dx}
% boundary integral
\newcommand{\bi}[3]{\int_{#3} {#1}{#2} \; ds}
% dt
\newcommand{\dt}[1]{\frac{d}{dt}{#1}}
%----------------------------------------------------------------------------------------
%	STRUCTURE MODIFICATIONS
%----------------------------------------------------------------------------------------

\setlength{\parskip}{3pt} % Slightly increase spacing between paragraphs

% Uncomment to center section titles
%\usepackage{sectsty}
%\sectionfont{\centering}

% Uncomment for Roman numerals for section numbers
%\renewcommand\thesection{\Roman{section}}

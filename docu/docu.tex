%----------------------------------------------------------------------------------------
%	PACKAGES AND OTHER DOCUMENT CONFIGURATIONS
%----------------------------------------------------------------------------------------

\documentclass[
12pt, % Default font size is 10pt, can alternatively be 11pt or 12pt
a4paper, % Alternatively letterpaper for US letter
onecolumn, % Alternatively twocolumn
portrait % Alternatively landscape
]{article}

\input{structure.tex} % Input the file specifying the document layout and structure

%----------------------------------------------------------------------------------------
%	ARTICLE INFORMATION
%----------------------------------------------------------------------------------------

\articletitle{Notes on the simulation of the heat equation in Firedrake} 

\datenotesstarted{October 10, 2017}
\docdate{\datenotesstarted; rev. \today}

\docauthor{Simon Pirkelmann}

%----------------------------------------------------------------------------------------

\begin{document}

\pagestyle{myheadings} % Use custom headers
\markright{\doctitle} % Place the article information into the header

%----------------------------------------------------------------------------------------
%	PRINT ARTICLE INFORMATION
%----------------------------------------------------------------------------------------

\thispagestyle{plain} % Plain formatting on the first page

\printtitle % Print the title

%----------------------------------------------------------------------------------------
%	ARTICLE NOTES
%----------------------------------------------------------------------------------------
\section{Setting}
Consider the following equation
\begin{equation}
y_t -  a \Delta y = 0 \text{ on } \Omega
\end{equation}
As a domain we consider the unit square. The boundary is partitioned in an isolating boundary $\Gamma_i$ and a control boundary $\Gamma_c$.
On the control part of the boundary we have the Dirichlet condition
\begin{equation}
y = u \text{ on } \Gamma_c.
\end{equation}
On the isolating part we have
\begin{equation}
- \frac{\partial y}{\partial n} = 0 \text{ on } \Gamma_i.
\end{equation}

For simplicity we use a single Robin type boundary condition instead
\begin{equation}
- \frac{\partial y}{\partial n} = \gamma (y - u) \text{ on } \Gamma.
\label{eq:robin-bc}
\end{equation}
By choosing 
$\begin{cases}
 \gamma = 0 &\text{ on } \Gamma_i \\ 
\gamma = 10^{6} &\text{ on } \Gamma_c \\
\end{cases}
$,
we can approximate both types of boundary conditions in a uniform way.
This will also allow us to extend the setting more easily in the future.
\section{Weak form}
For the weak for of the equation we replace $y_t$ by $\frac{y_{k+1} - y_k}{h}$ and $y$ by $y_{k+1}$ using backward Euler with sampling rate $h > 0$. Multiplying with a test function $v$ and integrating yields
\begin{equation}
\int_{\Omega} \frac{y_{k+1} - y_k}{h} v \; dx - a \int_{\Omega} \Delta y_{k+1} v \; dx = 0
\end{equation}
Using partial integration on the second integral we get
\begin{equation}
\int_{\Omega} \frac{y_{k+1} - y_k}{h} v \; dx - a \int_{\Gamma} \frac{\partial y_{k+1}}{\partial n} v \; ds + a \int_{\Omega} \nabla y_{k+1} \cdot \nabla v \; dx = 0
\end{equation}
Substituting the boundary condition \eqref{eq:robin-bc} we obtain
\begin{equation}
\int_{\Omega} \frac{y_{k+1} - y_k}{h} v \; dx + a \gamma \int_{\Gamma}  (y_{k+1} - u) v \; ds + a \int_{\Omega} \nabla y_{k+1} \cdot \nabla v \; dx = 0
\end{equation}
We can group the terms in the previous equations by terms that depend on $y_{k+1}$, and terms that do not depend on $y_{k+1}$ (an thus only depend on the external data ($y_k$ and $u$):
\begin{equation}
\underbrace{
\int_{\Omega} y_{k+1} v \; dx + h a \gamma \int_{\Gamma} y_{k+1} v \; ds + h a \int_{\Omega} \nabla y_{k+1} \cdot \nabla v \; dx
}_{a(y_{k+1}, v)}
 = 
\underbrace{ 
\int_{\Omega} y_k v \; dx + h a \gamma \int_{\Gamma} u v \; ds
}_{F(v)}
\end{equation}

\section{Implementation in Firedrake}
To solve the equation $a(y_{k+1}, v) = F(v)$ in Firedrake we first need to define the bilinearform (?) $a$ and the functional $F$.

\begin{python}
self.a = (self.y1 * v + h * ac * inner(grad(self.y1), grad(v))) * dx
for i in range(1,5):
    self.a += h * ac * Constant(gamma[i-1]) * self.y1 * v * ds(i)
\end{python}


\begin{python}
self.F = inner(self.y0, v) * dx
for i in range(1, 5):
    self.F += h * ac * Constant(gamma[i-1]) * Constant(u[i-1]) * v * ds(i)
\end{python}

\newpage
%----------------------------------------------------------------------------------------
%	BIBLIOGRAPHY
%----------------------------------------------------------------------------------------

\renewcommand{\refname}{Reference} % Change the default bibliography title

\bibliography{../bibtex} % Input your bibliography file
\bibliographystyle{plain}


%----------------------------------------------------------------------------------------

\end{document}

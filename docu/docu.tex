%----------------------------------------------------------------------------------------
%	PACKAGES AND OTHER DOCUMENT CONFIGURATIONS
%----------------------------------------------------------------------------------------

\documentclass[
12pt, % Default font size is 10pt, can alternatively be 11pt or 12pt
a4paper, % Alternatively letterpaper for US letter
onecolumn, % Alternatively twocolumn
portrait % Alternatively landscape
]{article}

%%%%%%%%%%%%%%%%%%%%%%%%%%%%%%%%%%%%%%%%%
% Article Notes
% Structure Specification File
% Version 1.0 (1/10/15)
%
% This file has been downloaded from:
% http://www.LaTeXTemplates.com
%
% Authors:
% Vel (vel@latextemplates.com)
% Christopher Eliot (christopher.eliot@hofstra.edu)
% Anthony Dardis (anthony.dardis@hofstra.edu)
%
% License:
% CC BY-NC-SA 3.0 (http://creativecommons.org/licenses/by-nc-sa/3.0/)
%
%%%%%%%%%%%%%%%%%%%%%%%%%%%%%%%%%%%%%%%%%

%----------------------------------------------------------------------------------------
%	REQUIRED PACKAGES
%----------------------------------------------------------------------------------------

\usepackage[includeheadfoot,columnsep=2cm, left=1in, right=1in, top=.5in, bottom=.5in]{geometry} % Margins

%\usepackage[T1]{fontenc} % For international characters
\usepackage[utf8]{inputenc}
%\usepackage{XCharter} % XCharter as the main font

\usepackage{natbib} % Use natbib to manage the reference
%\usepackage{cite}
%\bibliographystyle{apalike} % Citation style
%\bibliographystyle{te} % Citation style

\usepackage[english]{babel} % Use english by default
\usepackage{amsmath}
\usepackage{amssymb}
\usepackage{amsthm}
\usepackage{xcolor}
\usepackage{graphicx}
\usepackage{enumerate}
\usepackage{todonotes}
\usepackage{listings}
\usepackage{pythonhighlight}

\newtheorem{df}{Definition}
\newtheorem{ex}{Example}
\newtheorem{as}{Assumption}
\newtheorem{rem}{Remark}
\newtheorem{pr}{Proposition}
\newtheorem{qu}{Question}
\newtheorem{lm}{Lemma}
\newtheorem{thm}{Theorem}

%----------------------------------------------------------------------------------------
%	CUSTOM COMMANDS
%----------------------------------------------------------------------------------------

\newcommand{\articletitle}[1]{\renewcommand{\articletitle}{#1}} % Define a command for storing the article title
%\newcommand{\articlecitation}[1]{\renewcommand{\articlecitation}{#1}} % Define a command for storing the article citation
\newcommand{\doctitle}{``\articletitle''} % Define a command to store the article information as it will appear in the title and header

\newcommand{\datenotesstarted}[1]{\renewcommand{\datenotesstarted}{#1}} % Define a command to store the date when notes were first made
\newcommand{\docdate}[1]{\renewcommand{\docdate}{#1}} % Define a command to store the date line in the title

\newcommand{\docauthor}[1]{\renewcommand{\docauthor}{#1}} % Define a command for storing the article notes author

% Define a command for the structure of the document title
\newcommand{\printtitle}{
\begin{center}
\textbf{\Large{\doctitle}}

\docdate

\docauthor
\end{center}
}

%----------------------------------------------------------------------------------------
%	STRUCTURE MODIFICATIONS
%----------------------------------------------------------------------------------------

\setlength{\parskip}{3pt} % Slightly increase spacing between paragraphs

% Uncomment to center section titles
%\usepackage{sectsty}
%\sectionfont{\centering}

% Uncomment for Roman numerals for section numbers
%\renewcommand\thesection{\Roman{section}}
 % Input the file specifying the document layout and structure

%----------------------------------------------------------------------------------------
%	ARTICLE INFORMATION
%----------------------------------------------------------------------------------------

\articletitle{Notes on the simulation of the convection diffusion equation in FEniCS} 

\datenotesstarted{October 10, 2017}
\docdate{\datenotesstarted; rev. \today}

\docauthor{Simon Pirkelmann}

%----------------------------------------------------------------------------------------

\begin{document}

\pagestyle{myheadings} % Use custom headers
\markright{\doctitle} % Place the article information into the header

%----------------------------------------------------------------------------------------
%	PRINT ARTICLE INFORMATION
%----------------------------------------------------------------------------------------

\thispagestyle{plain} % Plain formatting on the first page

\printtitle % Print the title

%----------------------------------------------------------------------------------------
%	ARTICLE NOTES
%----------------------------------------------------------------------------------------
\section{Setting}
Consider the following equation
\begin{equation}
y_t -  \alpha \Delta y + w \nabla y = 0 \text{ on } Q := \Omega \times [0,T]
\label{eq:pde}
\end{equation}
where $y : Q \rightarrow \mathbb{R}$ is the temperature, $\alpha \in \mathbb{R}$ is the diffusion coefficient and $w : [0,T] \rightarrow \Omega$ is the velocity field. We use the shorthand $y_t = \frac{\partial y}{\partial t}$ to denote the time derivative.

Let $\Omega$ be the domain. The boundary is partitioned in an boundary $\Gamma_{out}$ where some outside temperature is prescribed and a control boundary $\Gamma_c$. On one part of the boundary a controllable t is applied which is described by a Dirichlet condition
\begin{equation}
y = u \text{ on } \Gamma_c.
\end{equation}
On the other part we have
\begin{equation}
y = y_{out} \text{ on } \Gamma_{out}
\end{equation}
where $\frac{\partial y}{\partial n}$ is the derivative of $y$ in normal direction.
We approximate the Dirichlet boundary conditions by using the following Robin type boundary condition instead
\begin{equation}
\frac{\partial y}{\partial n} + \gamma y = \gamma z \text{ on } \Gamma.
\label{eq:robin-bc}
\end{equation}
By choosing 
$\gamma := 10^3$ and
$z := \begin{cases}
y_{out} &\text{ on } \Gamma_{out} \\ 
u &\text{ on } \Gamma_c \\
\end{cases}
$
we can approximate the Dirichlet boundary conditions. This will also allow us to extend the setting more easily in the future.
\begin{center}
\begin{tikzpicture}
\draw[->] (-0.25,0) -- (3.25,0) node[anchor=north] {};
\draw[->] (0,-0.25) -- (0,3.25) {};

\draw [blue] plot coordinates {(0,0) (0,3) (3,3) (3,0)};

\draw [red] plot coordinates { (0,0) (3,0)};

\draw	(-0.2,-0.25) node {$0$}
		(-0.2,3) node {$1$}
		(3,-0.25) node {$1$}	
		(1.5, 1.5) node {\large{$\Omega$}}
		(1.5, -0.3) node[red] {\large{$\Gamma_c$}}
		(1.5, 3.28) node[blue] {\large{$\Gamma_i$}}
		;
\end{tikzpicture}
\end{center}
\section{Numerical simulation of the convection diffusion equation}
We simulate the equation using the finite element method.
\subsection{Weak form}
% linear functionals for boundary
\newcommand{\Ft}[1]{\langle F(t), {#1} \rangle}
\newcommand{\Bt}[2]{\langle B {#1}, {#2} \rangle}
\newcommand{\Fk}[1]{\langle F_k, {#1} \rangle}
% linear functionals for boundary (explicit)
\newcommand{\Fte}[2]{\alpha \go {#1} \bi{}{#2}{\Go}}
\newcommand{\Bte}[2]{\alpha \gc {#1} \bi{}{#2}{\Gc}}
% bilinear form a(t; phi, psi)
\newcommand{\at}[2]{a(t; {#1},{#2})}
% bilinear form a(t; phi, psi) (explicit)
\newcommand{\ate}[2]{\alpha \ltwospe{\nabla {#1}}{\nabla {#2}} + \int_{\Omega} (w \cdot \nabla {#1}) {#2} \; dx + \alpha\gc \bi{{#1}}{{#2}}{\Gc} + \alpha \go \bi{{#1}}{{#2}}{\Go}}

Multiplying with a test function and integrating over the domain $\Omega$ yields
\begin{equation}
\int_{\Omega} \frac{d}{dt}y v \; dx - \alpha \int_{\Omega} \Delta y v \; dx + \int_{\Omega} (w \cdot \nabla y) v \; dx  = 0
\end{equation}
Using partial integration and substituting the boundary conditions we obtain
\begin{equation}
\ltwospe{\dt{y}}{v} + \alpha \ltwospe{\nabla y}{\nabla v} + \ltwospe{(w \cdot \nabla y)}{v} + \alpha \gc \bi{(y - u)}{v}{\Gc} + \alpha \go \bi{(y -  \yo)}{v}{\Go} = 0
\end{equation}
We reorder by terms depending on $y$:
\begin{align*}
\ltwospe{\dt{y}}{v} + \alpha \ltwospe{\nabla y}{\nabla v} + \ltwospe{(w \cdot \nabla y)}{v} + \alpha \gc \bi{y}{v}{\Gc} + \alpha \go \bi{y}{v}{\Go} \\
= \alpha \gc \bi{u}{v}{\Gc} + \alpha \go \bi{\yo}{v}{\Go}
\end{align*}
This can be written more compactly be using the definitions 
\begin{align*}
\ltwosp{\varphi}{\psi} &= \ltwospe{\varphi}{\psi} \\
\Ft{\varphi} &= \Fte{\yo(t)}{\varphi} \\
\Bt{u}{\varphi} &= \Bte{u}{\varphi}
\end{align*}
and the bilinear form
\begin{equation}
\at{\varphi}{\psi} = \ate{\varphi}{\psi}
\end{equation}
We get
\begin{equation}
\ltwosp{\dt{y}}{v} + \at{y}{v} = \Bt{u}{v} + \Ft{v}
\label{eq:fem-system}
\end{equation}

\subsection{Time discretization}
We discretize in time using the implicit Euler method. We pick a sampling rate $h > 0$ and define $y_k := y(\cdot, t_0 + h k)$, $y_{out,k} := y_{out}(\cdot, t_0 + h k) $, $u_k := u(t_0 + h k)$, $z_k := z(t_0 + h k)$ for $k \in \{0, 1, \hdots, N\}$. \\
The time derivative of the state is approximated by
\begin{equation}
\dt y \approx \frac{y_{k+1} - y_k}{h}
\end{equation}
The initial value $y_0$ is given. To compute the next state $y_{k+1}$ for each $k \in \{0, 1, \hdots, N-1\}$  we replace $\dt{y}$ in equation \eqref{eq:fem-system} by $\frac{y_{k+1} - y_k}{h}$ and $y$ by $y_{k+1}$, as well as $\yo(t)$ by $\yo_{,k+1}$ and $u(t)$ by $u_{k}$. Note that we use $u_k$ instead of $u_{k+1}$ since we assume $u$ to be piecewise constant on each time interval.
This leads to 
\begin{equation}
\ltwosp{\frac{y_{k+1} - y_k}{h}}{v} + \at{y_{k+1}}{v} = \Bt{u_k}{v} + \Fk{v}
\end{equation}
or explicitely
\begin{align*}
\ltwospe{\frac{y_{k+1} - y_k}{h}}{v} + \ate{y_{k+1}}{v} \\= \Bte{u_k}{v} + \Fte{\yo_{,k+1}}{v}
\end{align*}
Again, reordering by the known and unknown terms yields
\begin{align*}
\ltwospe{\frac{y_{k+1}}{h}}{v} + \ate{y_{k+1}}{v} \\= \ltwospe{\frac{y_k}{h}}{v} + \Bte{u_k}{v} + \Fte{\yo_{,k+1}}{v}
\end{align*}

\subsection{Implementation in Firedrake}
The above variational equation is solved in FEniCs for each $k \in \{0, 1, \hdots, N-1\}$:
\begin{python}
# define a mesh
mesh = UnitIntervalMesh(50)

# Compile sub domains for boundaries
left = CompiledSubDomain("near(x[0], 0.)")
right = CompiledSubDomain("near(x[0], 1.)")

# Label boundaries, required for the objective
boundary_parts = MeshFunction("size_t", mesh, mesh.topology().dim() - 1)
left.mark(boundary_parts, 0)  # boundary part for outside temperature
right.mark(boundary_parts, 1)  # boundary part where control is applied
ds = Measure("ds", subdomain_data=boundary_parts)

# Choose a time step size
delta_t = 5.0e-3

# define constants of the PDE
alpha = Constant(1.0)
gamma = Constant(1.0e6)

U = FunctionSpace(mesh, "Lagrange", 1)
\end{python}

\begin{python}
# variational formulation
lhs = (y_k1 / Constant(delta_t) * phi) * dx + alpha * inner(grad(phi), grad(y_k1)) * dx + alpha * gamma * phi * y_k1 * ds
rhs = (y_k0 / Constant(delta_t) * phi) * dx + alpha * gamma * u * phi * ds(1) + alpha * gamma * y_out * phi * ds(0)
\end{python}


\begin{python}
F = inner(y0, v) * dx
for i in range(1, 5):
    F += h * ac * Constant(gamma[i-1]) * Constant(u[i-1]) * v * ds(i)
\end{python}

\section{Optimal Control Problem}
Now that the simulation is running we want to implement an optimal control problem on top of it. Our goal is to solve the following problem:
\begin{align*}
\min_{y,u} J(y, u) = & \frac{1}{2}\int_{\Omega} (y(x, T) - y_{\Omega}(T,x))^2 \; dx + \frac{1}{2} \int_{0}^{T} \int_{\Omega} (y(x,t) - y_{\Omega}(t,x))^2 \; dx \; dt \\
& + \frac{\lambda}{2} \int_{0}^{T} \int_{\Gamma} (u(x,t))^2 \; ds \; dt \\
\text{s.t.} & \eqref{eq:pde}, \eqref{eq:robin-bc} \\
& \underline{u}(x,t) \leq u(x,t) \leq \overline{u}(x,t) \\
& \underline{y}(x,t) \leq y(x,t) \leq \overline{y}(x,t)
\end{align*}
While the constraints on the control can be dealt with rather straightforwardly, the state constraints present some challenges. We will use Lavrentiev regularisation to handle the state constraints. The Lavrentiev regularisation replaces the state constraint by a mixed state-control constraint. For this we introduce an additional control variable $v : \Omega \times [0, \infty) \rightarrow \mathbb{R}$ defined on the whole domain. This control variable will also be penalized in the cost functional, and so the cost functional is modified to
\begin{align*}
\min_{u,v} J(y, u, v) = & \frac{1}{2}\int_{\Omega} (y(x, T) - y_{\Omega}(T,x))^2 \; dx + \frac{1}{2} \int_{0}^{T} \int_{\Omega} (y(x,t) - y_{\Omega}(t,x))^2 \; dx \; dt \\
& \frac{\sigma}{2} \int_{0}^{T} \int_{\Omega} (v(x,t))^2 \; dx \; dt 
+ \frac{\lambda}{2} \int_{0}^{T} \int_{\Gamma} (u(x,t))^2 \; ds \; dt
\end{align*}
The state constraint is replaced by the auxiliary control constraint
%\begin{equation}

%\end{equation}
\begin{alignat*}{2}
\underline{y}(x,t) &\leq y(x,t) + \varepsilon v(x,t) &&\leq \overline{y}(x,t) \\
\Leftrightarrow \underbrace{\frac{1}{\varepsilon}(\underline{y}(x,t) - y(x,t))}_{\underline{v}(x,t)} &\leq v(x,t) &&\leq \underbrace{\frac{1}{\varepsilon}(\overline{y}(x,t) - y(x,t))}_{\overline{v}(x,t)} \\
\end{alignat*}

This kind of optimal control problem (with both a control on the boundary and a control in the domain) is considered in [Tr\"oltzsch, p.221ff]. We look at the general cost functional
\begin{align*}
\min_{y,u,v} J(y, u, v) = & \int_{\Omega} \phi(x, y(T)) \; dx + \int \int_{Q} \varphi(x, t, y, v) \; dx \; dt \\
& + \int \int_{\Sigma} \psi(x,t,y, u) \; ds \; dt \\
\end{align*}
subject to
\begin{alignat*}{2}
y_t - \alpha \Delta y + d(x,t,y,v) &= 0 && \text{ in } Q \\
\partial_n y + b(x,t,y,u) &= 0 && \text{ in } \Sigma \\
y(\cdot,0) &= y_0 && \text{ on } \Omega \\
v_a(x,t) \leq v(x,t) &\leq v_b(x,t) &&\text{ in } Q \\
u_a(x,t) \leq u(x,t) &\leq u_b(x,t) &&\text{ in } \Sigma
\end{alignat*}

In our case we have the following identities:
\begin{align*}
\phi(x,y(x,T)) & = \frac{1}{2} (y(x,T) - y_{\Omega}(x,T))^2 \\
\phi_y(x,y(x,T)) & = y(x,T) - y_{\Omega}(x,T) \\
\\
\varphi(x,t, y(x,t), v(x,t)) & = \frac{1}{2} (y(x,t) - y_{\Omega}(x,t))^2 + \frac{\sigma}{2} (v(x,t))^2 \\
\varphi_y(x,t,y(x,t), v(x,t)) &= y(x,t) - y_{\Omega}(x,t) \\
\varphi_v(x,t,y(x,t), v(x,t)) &= \sigma v(x,t) \\
\\
\psi(x,t,y(x,t), u(x,t)) &= \frac{\lambda}{2} (u(x,t))^2 \\
\psi_y(x,t,y(x,t), u(x,t)) &= 0 \\
\psi_u(x,t,y(x,t), u(x,t)) &= \lambda u(x,t) \\
\\
d(x,t, y(x,t), v(x,t)) &= 0 \\
d_y(x,t, y(x,t), v(x,t)) &= 0 \\
d_v(x,t, y(x,t), v(x,t)) &= 0 \\
\\
b(x,t, y(x,t), u(x,t)) &= \frac{\gamma(x,t)}{\beta(x,t)} (y(x,t) - z(x,t)) \\
b_y(x,t, y(x,t), u(x,t)) &= \frac{\gamma(x,t)}{\beta(x,t)} \\
b_u(x,t, y(x,t), u(x,t)) &= -\frac{\gamma(x,t)}{\beta(x,t)}
\end{align*}

\section{Derivation of the adjoint system}
We introduce the adjoint states $p_1$ and $p_2$ to remove the PDE equality constraints. The Lagrangian for the problem is given by
\begin{align*}
\mathcal{L}(y,u,v,p_1, p_2) = & J(y,u,v) - \int_{0}^{T} \int_{\Omega} (\frac{\partial y}{\partial t} - \alpha \Delta y + d(x,t,y,v)) p_1 \; dx \; dt \\ & - \int_{0}^{T} \int_{\Gamma} (\partial_n y + b(x,t,y,u)) p_2 \; dx \; dt
\end{align*}
For optimality we need
\begin{alignat*}{2}
D_y \mathcal{L}(\bar{y},\bar{u},\bar{v},p_1, p_2) y & = 0, &\text{ for all } y \text{ with } y(\cdot, 0) = 0 \\
D_u \mathcal{L}(\bar{y},\bar{u},\bar{v},p_1, p_2) (u - \bar{u}) & \geq 0, &\text{ for all } u \in U_{ad} \\
D_v \mathcal{L}(\bar{y},\bar{u},\bar{v},p_1, p_2) (v - \bar{v}) & \geq 0, &\text{ for all } v \in V_{ad} \\
\end{alignat*}
In the following we compute $D_y \mathcal{L}$, $D_u \mathcal{L}$ and $D_v \mathcal{L}$. We start with $D_y \mathcal{L}$:
\begin{align*}
 D_y \mathcal{L}(y, u,v, p_1, p_2) h = & D_y J(y,u,v) h - \int_{0}^{T} \int_{\Omega} (\frac{\partial h}{\partial t} - \alpha \Delta h + d_y(x,t,y,v) h) p_1 \; dx \; dt \\ 
& - \int_{0}^{T} \int_{\Gamma} (\partial_n h + b_y(x,t,y,u) h) p_2 \; dx \; dt,
\end{align*} 
where for $D_y J(y,u,v)$ it holds that
\begin{equation}
D_y J(y,u,v) h = \int_{\Omega} \phi_y(x,y) h(T) \; dx + \int_{0}^{T} \int_{\Omega} \varphi_y(x,t,y, v) h \; dx \; dt 
+ \int_{0}^{T} \int_{\Gamma} \psi_y(x,t,y, u) h \; ds \; dt.
\end{equation}
First we consider the term
\begin{align*}
&- \int_{0}^{T} \int_{\Omega} (\frac{\partial h}{\partial t} - \alpha \Delta h + d_y(x,t,y,v) h) p_1 \; dx \; dt \\
& = - \int_{\Omega} \int_{0}^{T} \frac{\partial h}{\partial t} p_1 \; dx \; dt + \int_{0}^{T} \int_{\Omega} \alpha \Delta h p_1 \; dx \; dt - \int_{0}^{T} \int_{\Omega} d_y(x,t,y,v) h p_1 \; dx \; dt.
\end{align*}
Using partial integration (in time) for the first term and Green's second formula (in space) for the second term we obtain
\begin{align*}
&- \int_{\Omega} \int_{0}^{T} \frac{\partial h}{\partial t} p_1 \; dx \; dt + \alpha \int_{0}^{T} \int_{\Omega}  \Delta h p_1 \; dx \; dt - \int_{0}^{T} \int_{\Omega} d_y(x,t,y,v) h p_1 \; dx \; dt\\
= &- \int_{\Omega} \left[h p_1 \right]_0^T \; dx + \int_{\Omega} \int_{0}^{T}   \frac{\partial p_1}{\partial t} h \; dx \; dt + \alpha \int_{0}^{T} \int_{\Omega} \Delta p_1 h \; dx \; dt \\ 
&+ \alpha \int_{0}^{T} \int_{\Gamma}
p_1 \frac{\partial h}{\partial n} \; ds \; dt - \alpha \int_{0}^{T} \int_{\Gamma}
\frac{\partial p_1}{\partial n} h \; ds \; dt
 - \int_{0}^{T} \int_{\Omega} d_y(x,t,y,v) p_1 h \; dx \; dt \\
= &\int_{\Omega} p_1(0) h(0) \; dx - \int_{\Omega} p_1(T) h(T) \; dx +  \int_{0}^{T} \int_{\Omega} \frac{\partial p_1}{\partial t} h \; dx \; dt + \alpha \int_{0}^{T} \int_{\Omega} \Delta p_1 h \; dx \; dt \\ 
&+ \alpha \int_{0}^{T} \int_{\Gamma}
p_1 \frac{\partial h}{\partial n} \; ds \; dt - \alpha \int_{0}^{T} \int_{\Gamma}
\frac{\partial p_1}{\partial n} h \; ds \; dt
 - \int_{0}^{T} \int_{\Omega} d_y(x,t,y,v) p_1 h \; dx \; dt \\
 \overset{h(0)=0}{=} & - \int_{\Omega} p_1(T) h(T) \; dx + \int_{\Omega} \int_{0}^{T}   \frac{\partial p_1}{\partial t} h \; dx \; dt + \alpha \int_{0}^{T} \int_{\Omega} \Delta p_1 h \; dx \; dt \\ 
&+ \alpha \int_{0}^{T} \int_{\Gamma}
p_1 \frac{\partial h}{\partial n} \; ds \; dt - \alpha \int_{0}^{T} \int_{\Gamma} \frac{\partial p_1}{\partial n} h \; ds \; dt
 - \int_{0}^{T} \int_{\Omega} d_y(x,t,y,v) p_1 h \; dx \; dt \\
\end{align*}
In the last step we used that $h(0) = 0$. This is explained in [Tr\"oltzsch, p.96] and follows from a substitution $y := y - \bar{y}$ for the state.\\
By substituting the above equation in the original equation and ordering by integration domains we obtain
\begin{align*}
D_y J(y,u,v) h = & \int_{\Omega} (\phi_y(x,y) - p_1(T)) h(T) \; dx \\
& + \int_{0}^{T} \int_{\Omega} (\frac{\partial p_1}{\partial t} + \alpha \Delta p_1 - d_y(x,t,y,v) p_1 + \varphi_y(x,t,y, v)) h \; dx \; dt  \\
& + \int_{0}^{T} \int_{\Gamma} (\alpha p_1 - p_2) \frac{\partial h}{\partial n} \; ds \; dt \\
& + \int_{0}^{T} \int_{\Gamma} (- \alpha \frac{\partial p_1}{\partial n} - b_y(x,t,y,u) p_2 + \psi_y(x,t,y, u))  h \; ds \; dt
\end{align*}
Now we make special choices of $h$, first $h \in C_0^\infty(Q)$, then with arbitrary $h(T)$, arbitrary $h|_\Sigma$  and finally with arbitrary $\frac{\partial h}{\partial n}$. From this we obtain the following adjoint equation:
\begin{equation}
\begin{alignedat}{2}
-p_{1,t} - \alpha \Delta p_1 + d_y(x,t,\bar{y}, \bar{v}) p_1 &= \varphi_y(x,t,\bar{y}, \bar{v}) &&\text{ in } Q \\
\alpha \partial_n p_1 + b_y(x,t,\bar{y}, \bar{u}) p_2 &= \psi_y(x,t,\bar{y}, \bar{u}) &&\text{ in } \Sigma \\
\alpha p_1 &= p_2 &&\text{ in } \Sigma \\
p_1(x,T) &= \phi_y(x, \bar{y}(x,T)) &&\text{ in } \Omega
\label{eq:adjoint-pde-preliminary}
\end{alignedat}
\end{equation}
By setting $p := p_1$, $p_2 = \alpha p$ we obtain:
\begin{equation}
\begin{alignedat}{2}
-p_t - \alpha \Delta p + d_y(x,t,\bar{y}, \bar{v}) p &= \varphi_y(x,t,\bar{y}, \bar{v}) &&\text{ in } Q \\
\partial_n p + b_y(x,t,\bar{y}, \bar{u}) p &= \frac{1}{\alpha}\psi_y(x,t,\bar{y}, \bar{u}) &&\text{ in } \Sigma \\
p(x,T) &= \phi_y(x, \bar{y}(x,T)) &&\text{ in } \Omega
\label{eq:adjoint-pde}
\end{alignedat}
\end{equation}
A similar result (but without the diffusion coefficient $\alpha$) can be found in [Tr\"oltzsch, p.225f].\\
Finally we also compute $D_u \mathcal{L}$ and $D_v \mathcal{L}$:
\begin{align*}
D_u \mathcal{L}(y,u,v, p) h & = D_u J(y,u,v) h - \int_{0}^{T} \int_{\Gamma} b_u(x,t,y,u) p h \; ds \; dt \\
& = \int_{0}^{T} \int_{\Gamma}  (\psi_u(x,t,y,u) -  b_u(x,t,y,u) p)  h \; ds \; dt
\end{align*}
\begin{align*}
D_v \mathcal{L}(y,u,v, p) h & = D_v J(y,u,v) h - \int_{0}^{T} \int_{\Omega} d_v(x,t,y,v) p h \; dx \; dt \\
& = \int_{0}^{T} \int_{\Omega}  (\varphi_v(x,t,y,v) -  d_v(x,t,y,v) p)  h \; dx \; dt
\end{align*}

\section{Weak form of the adjoint system}
To derive the weak form of the adjoint system \eqref{eq:adjoint-pde} we first note that the equation evolves backward in time. To get a forward equation we use the substitution $\tau := T - t$.
Define
\begin{align*}
\tilde{p}(x,\tau) &:= p(x,T-\tau) = p(x,t) \\
\tilde{y}(x,\tau) &:= y(x,T-\tau) \\
\tilde{u}(x,\tau) &:= u(x,T-\tau) \\
\tilde{v}(x,\tau) &:= v(x,T-\tau)
\end{align*}
Noting that $D_{\tau}\tilde{p}(x,\tau) = -D_t p(x,t)$, the adjoint system changes to
\begin{equation}
\begin{alignedat}{2}
\tilde{p}_{\tau} - \alpha \Delta \tilde{p} + d_y(x,T-\tau,\tilde{y}, \tilde{v}) \tilde{p} &= \varphi_y(x,T-\tau,\tilde{y}, \tilde{v}) &&\text{ in } Q \\
\partial_n \tilde{p} + b_y(x,T-\tau,\tilde{y}, \tilde{u}) \tilde{p} &= \frac{1}{\alpha}\psi_y(x,T-\tau,\tilde{y}, \tilde{u}) &&\text{ in } \Sigma \\
\tilde{p}(x,0) &= \phi_y(x, \tilde{y}(x,0)) &&\text{ in } \Omega
\label{eq:adjoint-pde-time-substitution}
\end{alignedat}
\end{equation}
Now the system can be solved again using backward Euler. Replace $\tilde{p}_\tau$ by $\frac{\tilde{p}_{k+1} - \tilde{p}_k}{h}$ and $\tilde{p}$, $\tilde{y}$, $\tilde{u}$ and $\tilde{v}$ by their discrete counterparts.  We multiply with the test function $v$ and integrate:
\begin{align*}
\int_{\Omega} \frac{\tilde{p}_{k+1} - \tilde{p}_k}{h} v \; dx - \int_{\Omega} \alpha \Delta \tilde{p}_{k+1} v \; dx + \int_{\Omega} d_y(x,T-\tau_{k}, \tilde{y}_k, \tilde{v}_k) \tilde{p}_{k+1} v \; dx = \int_{\Omega} \varphi_y(x,T-\tau_k,\tilde{y}_k, \tilde{v}_k) v \; dx
\end{align*}
Using partial integration on the second integral and inserting the boundary condition yields 
\begin{align*}
\int_{\Omega} \alpha \Delta \tilde{p}_{k+1} v \; dx & = - \int_{\Gamma} \alpha \frac{\partial \tilde{p}_{k+1}}{\partial n} v ds + \alpha \int_{\Omega}  \nabla \tilde{p}_{k+1} \nabla v \; dx \\
& = \int_{\Gamma} \alpha b_y(x,T - \tau_k, \tilde{y}_k, \tilde{u}_k) \tilde{p}_{k+1} v \; ds - \int_{\Gamma} \psi_y(x,T- \tau_k, \tilde{y}_k, \tilde{u}_k) v \; ds \\
&+ \alpha \int_{\Omega} \nabla \tilde{p}_{k+1} \nabla v \; dx
\end{align*}
We insert this in the first equation again and order by integration domain
\begin{align*}
\int_{\Omega} (\frac{\tilde{p}_{k+1} - \tilde{p}_k}{h} + d_y(x,T - \tau_{k}, \tilde{y}_k, \tilde{v}_k) \tilde{p}_{k+1} - \varphi_y(x,T - \tau_k,\tilde{y}_k, \tilde{v}_k)) v + \alpha \nabla \tilde{p}_{k+1} \nabla v dx & \\
+ \int_{\Gamma} (\alpha b_y(x,T-\tau_k, \tilde{y}_k, \tilde{u}) \tilde{p}_{k+1} - \psi_y(x,T-\tau_k, \tilde{y}_k, \tilde{u}_k) ) v \; ds  & = 0
\end{align*}
After the modified adjoint system was solved we get the original adjoint by substituting back $p(x,t_k) = \tilde{p}(x,\tau_k)$.

\subsection{Adjoint equation for our case}
In our case the equation simplifies to:
\begin{align*}
\int_{\Omega} (\frac{\tilde{p}_{k+1} - \tilde{p}_k}{h} - (\tilde{y}_k - \tilde{y}_{\Omega,k})) v + \alpha \nabla \tilde{p}_{k+1} \nabla v \; dx 
+ \int_{\Gamma} \frac{\alpha \gamma}{\beta} \tilde{p}_{k+1} v \; ds  = 0
\end{align*}

Additionally, we have the initial condition:
\begin{alignat*}{2}
\tilde{p}_{0} &= &\phi_y(x, \tilde{y}(x,0)) = y_{N} - y_{\Omega, N}
\end{alignat*}

Rewriting (without the time substitution for $y$) and inserting the different boundaries $\Gamma_{out}$ and $\Gamma_c$:
\begin{alignat*}{3}
&&\int_{\Omega} (\frac{\tilde{p}_{k+1} - \tilde{p}_k}{h} - (y_{N-k} - y_{\Omega,N-k})) v + \alpha \nabla \tilde{p}_{k+1} \nabla v \; dx && \\
&&+ \alpha \int_{\Gamma_{out}} \frac{\gamma_{out}}{\beta} \tilde{p}_{k+1} v \; ds + \alpha \int_{\Gamma_{c}} \frac{\gamma_{c}}{\beta} \tilde{p}_{k+1} v \; ds  = 0, && \\
&& && \text{ for } k \in \{0, \hdots, N-1\}.
\end{alignat*}

\section{Solution by Projected Gradient Method}
Let $N \in \mathbb{N}$ be the MPC horizon and let $u^n := (u_0^n, u_1^n, \hdots, u_{N-1}^n)$, $v^n := (v_0^n, v_1^n, \hdots, v_{N-1}^n)$ be the iterates of the optimization algorithm. \\ 
The gradient of the reduced cost functional $f(v,u) = J(y(v,u), v, u)$ is given by
\begin{align*}
f'(v^n, u^n)(v,u) = & \int \int_Q (\varphi_v(x,t, y^n, v^n) - d_v(x,t,y^n,v^n) p^n) \; v \; dx \; dt \\
& + \int \int_{\Sigma} (\psi_u(x,t,y^n, u^n) - b_u(x,t,y^n,u^n) p^n) \; u \; ds \; dt
\end{align*}
This can also be found in [Tr\"oltzsch, p. 243f], for the special case of $d(x,t,y,v) = v$, $b(x,t,y,u) = u$.

Solution algorithm:\\
\begin{enumerate}
\item Solve forward system for given $(u^n, v^n)$ $\rightsquigarrow y_n$ \\
\item Solve adjoint system $\rightsquigarrow p^n$ \\
\item Descent directions
\begin{align*}
&h^n := -(\varphi_v(\cdot, y^n, v^n) - d_v(\cdot,y^n,v^n) p^n) \\ 
&r^n := -(\psi_u(\cdot, y^n |_{\Sigma}, u^n) - b_u(\cdot,y^n,u^n) p^n |_{\Sigma})
\end{align*}
\item Compute step size $\rightsquigarrow s^n$
(e.g. use $\min_{s > 0} f(\mathbb{P}_V(v^n+s h^n), \mathbb{P}_U(u^n + s r^n)).$
\item New iterates:
\begin{equation}
(v^{n+1}, u^{n+1}) := (\mathbb{P}_V(v^n + s^n h^n), \mathbb{P}_U(u^n + s^n r^n))
\end{equation}
\end{enumerate}

\subsection{Gradient in our case}
\begin{align*}
f'(v^n, u^n)(v,u) = & \int \int_Q (\sigma v^n) \; v \; dx \; dt \\
& + \int \int_{\Sigma} (\lambda u^n) + \frac{\gamma}{\beta} p^n) \; u \; ds \; dt
\end{align*}

\section{Solver Options/Preconditioning}
\newpage
%----------------------------------------------------------------------------------------
%	BIBLIOGRAPHY
%----------------------------------------------------------------------------------------

\renewcommand{\refname}{Reference} % Change the default bibliography title

\bibliography{../bibtex} % Input your bibliography file
\bibliographystyle{plain}


%----------------------------------------------------------------------------------------

\end{document}

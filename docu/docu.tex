%----------------------------------------------------------------------------------------
%	PACKAGES AND OTHER DOCUMENT CONFIGURATIONS
%----------------------------------------------------------------------------------------

\documentclass[
12pt, % Default font size is 10pt, can alternatively be 11pt or 12pt
a4paper, % Alternatively letterpaper for US letter
onecolumn, % Alternatively twocolumn
portrait % Alternatively landscape
]{article}

%%%%%%%%%%%%%%%%%%%%%%%%%%%%%%%%%%%%%%%%%
% Article Notes
% Structure Specification File
% Version 1.0 (1/10/15)
%
% This file has been downloaded from:
% http://www.LaTeXTemplates.com
%
% Authors:
% Vel (vel@latextemplates.com)
% Christopher Eliot (christopher.eliot@hofstra.edu)
% Anthony Dardis (anthony.dardis@hofstra.edu)
%
% License:
% CC BY-NC-SA 3.0 (http://creativecommons.org/licenses/by-nc-sa/3.0/)
%
%%%%%%%%%%%%%%%%%%%%%%%%%%%%%%%%%%%%%%%%%

%----------------------------------------------------------------------------------------
%	REQUIRED PACKAGES
%----------------------------------------------------------------------------------------

\usepackage[includeheadfoot,columnsep=2cm, left=1in, right=1in, top=.5in, bottom=.5in]{geometry} % Margins

%\usepackage[T1]{fontenc} % For international characters
\usepackage[utf8]{inputenc}
%\usepackage{XCharter} % XCharter as the main font

\usepackage{natbib} % Use natbib to manage the reference
%\usepackage{cite}
%\bibliographystyle{apalike} % Citation style
%\bibliographystyle{te} % Citation style

\usepackage[english]{babel} % Use english by default
\usepackage{amsmath}
\usepackage{amssymb}
\usepackage{amsthm}
\usepackage{xcolor}
\usepackage{graphicx}
\usepackage{enumerate}
\usepackage{todonotes}
\usepackage{listings}
\usepackage{pythonhighlight}

\newtheorem{df}{Definition}
\newtheorem{ex}{Example}
\newtheorem{as}{Assumption}
\newtheorem{rem}{Remark}
\newtheorem{pr}{Proposition}
\newtheorem{qu}{Question}
\newtheorem{lm}{Lemma}
\newtheorem{thm}{Theorem}

%----------------------------------------------------------------------------------------
%	CUSTOM COMMANDS
%----------------------------------------------------------------------------------------

\newcommand{\articletitle}[1]{\renewcommand{\articletitle}{#1}} % Define a command for storing the article title
%\newcommand{\articlecitation}[1]{\renewcommand{\articlecitation}{#1}} % Define a command for storing the article citation
\newcommand{\doctitle}{``\articletitle''} % Define a command to store the article information as it will appear in the title and header

\newcommand{\datenotesstarted}[1]{\renewcommand{\datenotesstarted}{#1}} % Define a command to store the date when notes were first made
\newcommand{\docdate}[1]{\renewcommand{\docdate}{#1}} % Define a command to store the date line in the title

\newcommand{\docauthor}[1]{\renewcommand{\docauthor}{#1}} % Define a command for storing the article notes author

% Define a command for the structure of the document title
\newcommand{\printtitle}{
\begin{center}
\textbf{\Large{\doctitle}}

\docdate

\docauthor
\end{center}
}

%----------------------------------------------------------------------------------------
%	STRUCTURE MODIFICATIONS
%----------------------------------------------------------------------------------------

\setlength{\parskip}{3pt} % Slightly increase spacing between paragraphs

% Uncomment to center section titles
%\usepackage{sectsty}
%\sectionfont{\centering}

% Uncomment for Roman numerals for section numbers
%\renewcommand\thesection{\Roman{section}}
 % Input the file specifying the document layout and structure

%----------------------------------------------------------------------------------------
%	ARTICLE INFORMATION
%----------------------------------------------------------------------------------------

\articletitle{Notes on the simulation of the heat equation in Firedrake} 

\datenotesstarted{October 10, 2017}
\docdate{\datenotesstarted; rev. \today}

\docauthor{Simon Pirkelmann}

%----------------------------------------------------------------------------------------

\begin{document}

\pagestyle{myheadings} % Use custom headers
\markright{\doctitle} % Place the article information into the header

%----------------------------------------------------------------------------------------
%	PRINT ARTICLE INFORMATION
%----------------------------------------------------------------------------------------

\thispagestyle{plain} % Plain formatting on the first page

\printtitle % Print the title

%----------------------------------------------------------------------------------------
%	ARTICLE NOTES
%----------------------------------------------------------------------------------------
\section{Setting}
Consider the following equation
\begin{equation}
y_t -  a \Delta y = 0 \text{ on } \Omega
\label{eq:pde}
\end{equation}
where $y : \Omega \times [0, \infty) \rightarrow \mathbb{R}$ is the temperature, $a \in \mathbb{R}$ is the diffusion coefficient. We use the shorthand $y_t = \frac{\partial y}{\partial t}$ to denote the time derivative.
 
As a domain we consider the unit square. The boundary is partitioned in an isolating boundary $\Gamma_i$ and a control boundary $\Gamma_c$.
On the control part of the boundary heating/cooling is applied which is described by a Dirichlet condition
\begin{equation}
y = u \text{ on } \Gamma_c.
\end{equation}
On the isolating part we have
\begin{equation}
- \frac{\partial y}{\partial n} = 0 \text{ on } \Gamma_i
\end{equation}
where $\frac{\partial y}{\partial n}$ is the derivative of $y$ in normal direction.

\begin{center}
\begin{tikzpicture}
\draw[->] (-0.25,0) -- (3.25,0) node[anchor=north] {};
\draw[->] (0,-0.25) -- (0,3.25) {};

\draw [blue] plot coordinates {(0,0) (0,3) (3,3) (3,0)};

\draw [red] plot coordinates { (0,0) (3,0)};

\draw	(-0.2,-0.25) node {$0$}
		(-0.2,3) node {$1$}
		(3,-0.25) node {$1$}	
		(1.5, 1.5) node {\large{$\Omega$}}
		(1.5, -0.3) node[red] {\large{$\Gamma_c$}}
		(1.5, 3.28) node[blue] {\large{$\Gamma_i$}}
		;
\end{tikzpicture}
\end{center}


For simplicity we use a single Robin type boundary condition instead
\begin{equation}
- \frac{\partial y}{\partial n} = \gamma (y - u) \text{ on } \Gamma.
\label{eq:robin-bc}
\end{equation}
By choosing 
$\begin{cases}
 \gamma = 0 &\text{ on } \Gamma_i \\ 
\gamma = 10^{3} &\text{ on } \Gamma_c \\
\end{cases}
$,
we can approximate both types of boundary conditions in a uniform way.
This will also allow us to extend the setting more easily in the future.
\section{Weak Form}
For the weak for of the equation we replace $y_t$ by $\frac{y_{k+1} - y_k}{h}$ and $y$ by $y_{k+1}$ using backward Euler with sampling rate $h > 0$. Multiplying with a test function $v$ and integrating yields
\begin{equation}
\int_{\Omega} \frac{y_{k+1} - y_k}{h} v \; dx - a \int_{\Omega} \Delta y_{k+1} v \; dx = 0
\end{equation}
Using partial integration on the second integral we get
\begin{equation}
\int_{\Omega} \frac{y_{k+1} - y_k}{h} v \; dx - a \int_{\Gamma} \frac{\partial y_{k+1}}{\partial n} v \; ds + a \int_{\Omega} \nabla y_{k+1} \cdot \nabla v \; dx = 0
\end{equation}
Substituting the boundary condition \eqref{eq:robin-bc} we obtain
\begin{equation}
\int_{\Omega} \frac{y_{k+1} - y_k}{h} v \; dx + a \gamma \int_{\Gamma}  (y_{k+1} - u) v \; ds + a \int_{\Omega} \nabla y_{k+1} \cdot \nabla v \; dx = 0
\end{equation}
We can group the terms in the previous equations by terms that depend on $y_{k+1}$, and terms that do not depend on $y_{k+1}$ (an thus only depend on the external data ($y_k$ and $u$):
\begin{equation}
\underbrace{
\int_{\Omega} y_{k+1} v \; dx + h a \gamma \int_{\Gamma} y_{k+1} v \; ds + h a \int_{\Omega} \nabla y_{k+1} \cdot \nabla v \; dx
}_{a(y_{k+1}, v)}
 = 
\underbrace{ 
\int_{\Omega} y_k v \; dx + h a \gamma \int_{\Gamma} u v \; ds
}_{F(v)}
\end{equation}

\section{Implementation in Firedrake}
To solve the equation $a(y_{k+1}, v) = F(v)$ in Firedrake we first need to define the bilinearform (?) $a$ and the functional $F$.

\begin{python}
# Definitions 
S = FunctionSpace(self.mesh, "CG", 1)
v = TestFunction(S)
y1 = Function(S)
y0 = Function(S)

h = Constant(0.001)
gamma = [0.0, 0.0, 1.0e3, 0.0] # gamma for different parts of the the boundary
ac = 0.2
u = [0.0, 0.0, 25.0, 0.0]
\end{python}

\begin{python}
a = (y1 * v + h * ac * inner(grad(y1), grad(v))) * dx
for i in range(1,5):
    a += h * ac * Constant(gamma[i-1]) * y1 * v * ds(i)
\end{python}


\begin{python}
F = inner(y0, v) * dx
for i in range(1, 5):
    F += h * ac * Constant(gamma[i-1]) * Constant(u[i-1]) * v * ds(i)
\end{python}

\section{Optimal Control Problem}
Now that the simulation is running we want to implement an optimal control problem on top of it. Our goal is to solve the following problem:
\begin{align*}
\min_{y,u} J(y, u) = & \frac{1}{2}\int_{\Omega} (y(x, T) - y_{\Omega}(T,x))^2 \; dx + \frac{1}{2} \int_{0}^{T} \int_{\Omega} (y(x,t) - y_{\Omega}(t,x))^2 \; dx \; dt \\
& + \frac{\lambda}{2} \int_{0}^{T} \int_{\Gamma} (u(x,t))^2 \; ds \; dt \\
\text{s.t.} & \eqref{eq:pde}, \eqref{eq:robin-bc} \\
& \underline{u}(x,t) \leq u(x,t) \leq \overline{u}(x,t) \\
& \underline{y}(x,t) \leq y(x,t) \leq \overline{y}(x,t)
\end{align*}
While the constraints on the control can be dealt with rather straightforwardly, the state constraints present some challenges. We will use Lavrentiev regularisation to handle the state constraints. The Lavrentiev regularisation replaces the state constraint by a mixed state-control constraint. For this we introduce an additional control variable $v : \Omega \times [0, \infty) \rightarrow \mathbb{R}$ defined on the whole domain. This control variable will also be penalized in the cost functional, and so the cost functional is modified to
\begin{align*}
\min_{u,v} J(y, u, v) = & \frac{1}{2}\int_{\Omega} (y(x, T) - y_{\Omega}(T,x))^2 \; dx + \frac{1}{2} \int_{0}^{T} \int_{\Omega} (y(x,t) - y_{\Omega}(t,x))^2 \; dx \; dt \\
& \frac{\sigma}{2} \int_{0}^{T} \int_{\Omega} (v(x,t))^2 \; dx \; dt 
+ \frac{\lambda}{2} \int_{0}^{T} \int_{\Gamma} (u(x,t))^2 \; ds \; dt
\end{align*}
The state constraint is replaced by the auxiliary control constraint
%\begin{equation}

%\end{equation}
\begin{alignat*}{2}
\underline{y}(x,t) &\leq y(x,t) + \varepsilon v(x,t) &&\leq \overline{y}(x,t) \\
\Leftrightarrow \underbrace{\frac{1}{\varepsilon}(\underline{y}(x,t) - y(x,t))}_{\underline{v}(x,t)} &\leq v(x,t) &&\leq \underbrace{\frac{1}{\varepsilon}(\overline{y}(x,t) - y(x,t))}_{\overline{v}(x,t)} \\
\end{alignat*}

This kind of optimal control problem (with both a control on the boundary and a control in the domain) is considered in [Tr\"oltzsch, p.221ff]. We look at the general cost functional
\begin{align*}
\min_{y,u,v} J(y, u, v) = & \int_{\Omega} \phi(x, y(T)) \; dx + \int \int_{Q} \varphi(x, t, y, v) \; dx \; dt \\
& + \int \int_{\Sigma} \psi(x,t,y, u) \; ds \; dt \\
\end{align*}
subject to
\begin{alignat*}{2}
y_t - \Delta y + d(x,t,y,v) &= 0 && \text{ in } Q \\
\partial_n y + b(x,t,y,u) &= 0 && \text{ in } \Sigma \\
y(\cdot,0) &= y_0 && \text{ on } \Omega \\
v_a(x,t) \leq v(x,t) &\leq v_b(x,t) &&\text{ in } Q \\
u_a(x,t) \leq u(x,t) &\leq u_b(x,t) &&\text{ in } \Sigma
\end{alignat*}

In our case we have the following identities:
\begin{align*}
\phi(x,y(x,T)) & = \frac{1}{2} (y(x,T) - y_{\Omega}(x,T))^2 \\
\phi_y(x,y(x,T)) & = y(x,T) - y_{\Omega}(x,T) \\
\\
\varphi(x,t, y(x,t), v(x,t)) & = \frac{1}{2} (y(x,t) - y_{\Omega}(x,t))^2 + \frac{\sigma}{2} (v(x,t))^2 \\
\varphi_y(x,t,y(x,t), v(x,t)) &= y(x,t) - y_{\Omega}(x,t) \\
\varphi_v(x,t,y(x,t), v(x,t)) &= \sigma v(x,t) \\
\\
\psi(x,t,y(x,t), u(x,t)) &= \frac{\lambda}{2} (u(x,t))^2 \\
\psi_y(x,t,y(x,t), u(x,t)) &= 0 \\
\psi_u(x,t,y(x,t), u(x,t)) &= \lambda u(x,t) \\
\\
d(x,t, y(x,t), v(x,t)) &= 0 \\
d_y(x,t, y(x,t), v(x,t)) &= 0 \\
d_v(x,t, y(x,t), v(x,t)) &= 0 \\
\\
b(x,t, y(x,t), u(x,t)) &= \gamma(x,t) (y(x,t) - u(x,t)) \\
b_y(x,t, y(x,t), u(x,t)) &= \gamma(x,t) \\
b_u(x,t, y(x,t), u(x,t)) &= -\gamma(x,t)
\end{align*}

\section{Derivation of the adjoint system}
We introduce the adjoint states $p_1$ and $p_2$ to remove the PDE equality constraints. The Lagrangian for the problem is given by
\begin{align*}
\mathcal{L}(y,u,v,p_1, p_2) = & J(y,u,v) - \int_{0}^{T} \int_{\Omega} (\frac{\partial y}{\partial t} - \Delta y + d(x,t,y,v)) p_1 \; dx \; dt \\ & - \int_{0}^{T} \int_{\Gamma} (\partial_n y + b(x,t,y,u)) p_2 \; dx \; dt
\end{align*}
For optimality we need
\begin{alignat*}{2}
D_y \mathcal{L}(\bar{y},\bar{u},\bar{v},p_1, p_2) y & = 0, &\text{ for all } y \text{ with } y(\cdot, 0) = 0 \\
D_u \mathcal{L}(\bar{y},\bar{u},\bar{v},p_1, p_2) (u - \bar{u}) & \geq 0, &\text{ for all } u \in U_{ad} \\
D_v \mathcal{L}(\bar{y},\bar{u},\bar{v},p_1, p_2) (v - \bar{v}) & \geq 0, &\text{ for all } v \in V_{ad} \\
\end{alignat*}
In the following we compute $D_y \mathcal{L}$, $D_u \mathcal{L}$ and $D_v \mathcal{L}$. We start with $D_y \mathcal{L}$:
\begin{align*}
 D_y \mathcal{L}(y, u,v, p_1, p_2) h = & D_y J(y,u,v) h - \int_{0}^{T} \int_{\Omega} (\frac{\partial h}{\partial t} - \Delta h + d_y(x,t,y,v) h) p_1 \; dx \; dt \\ 
& - \int_{0}^{T} \int_{\Gamma} (\partial_n h + b_y(x,t,y,u) h) p_2 \; dx \; dt,
\end{align*} 
where for $D_y J(y,u,v)$ it holds that
\begin{equation}
D_y J(y,u,v) h = \int_{\Omega} \phi_y(x,y) h(T) \; dx + \int_{0}^{T} \int_{\Omega} \varphi_y(x,t,y, v) h \; dx \; dt 
+ \int_{0}^{T} \int_{\Gamma} \psi_y(x,t,y, u) h \; ds \; dt.
\end{equation}
First we consider the term
\begin{align*}
&- \int_{0}^{T} \int_{\Omega} (\frac{\partial h}{\partial t} - \Delta h + d_y(x,t,y,v) h) p_1 \; dx \; dt \\
& = - \int_{\Omega} \int_{0}^{T} \frac{\partial h}{\partial t} p_1 \; dx \; dt + \int_{0}^{T} \int_{\Omega} \Delta h p_1 \; dx \; dt - \int_{0}^{T} \int_{\Omega} d_y(x,t,y,v) h p_1 \; dx \; dt.
\end{align*}
Using partial integration (in time) for the first term and Green's second formula (in space) for the second term we obtain
\begin{align*}
&- \int_{\Omega} \int_{0}^{T} \frac{\partial h}{\partial t} p_1 \; dx \; dt + \int_{0}^{T} \int_{\Omega} \Delta h p_1 \; dx \; dt - \int_{0}^{T} \int_{\Omega} d_y(x,t,y,v) h p_1 \; dx \; dt\\
= &- \int_{\Omega} \left[h p_1 \right]_0^T \; dx + \int_{\Omega} \int_{0}^{T}   \frac{\partial p_1}{\partial t} h \; dx \; dt + \int_{0}^{T} \int_{\Omega} \Delta p_1 h \; dx \; dt \\ 
&+ \int_{0}^{T} \int_{\Gamma}
p_1 \frac{\partial h}{\partial n} \; ds \; dt - \int_{0}^{T} \int_{\Gamma}
\frac{\partial p_1}{\partial n} h \; ds \; dt
 - \int_{0}^{T} \int_{\Omega} d_y(x,t,y,v) p_1 h \; dx \; dt \\
= &\int_{\Omega} p_1(0) h(0) \; dx - \int_{\Omega} p_1(T) h(T) \; dx +  \int_{0}^{T} \int_{\Omega} \frac{\partial p_1}{\partial t} h \; dx \; dt + \int_{0}^{T} \int_{\Omega} \Delta p_1 h \; dx \; dt \\ 
&+ \int_{0}^{T} \int_{\Gamma}
p_1 \frac{\partial h}{\partial n} \; ds \; dt - \int_{0}^{T} \int_{\Gamma}
\frac{\partial p_1}{\partial n} h \; ds \; dt
 - \int_{0}^{T} \int_{\Omega} d_y(x,t,y,v) p_1 h \; dx \; dt \\
 \overset{h(0)=0}{=} & - \int_{\Omega} p_1(T) h(T) \; dx + \int_{\Omega} \int_{0}^{T}   \frac{\partial p_1}{\partial t} h \; dx \; dt + \int_{0}^{T} \int_{\Omega} \Delta p_1 h \; dx \; dt \\ 
&+ \int_{0}^{T} \int_{\Gamma}
p_1 \frac{\partial h}{\partial n} \; ds \; dt - \int_{0}^{T} \int_{\Gamma} \frac{\partial p_1}{\partial n} h \; ds \; dt
 - \int_{0}^{T} \int_{\Omega} d_y(x,t,y,v) p_1 h \; dx \; dt \\
\end{align*}
In the last step we used that $h(0) = 0$. This is explained in [Tr\"oltzsch, p.96] and follows from a substitution $y := y - \bar{y}$ for the state.\\
By substituting the above equation in the original equation and ordering by integration domains we obtain
\begin{align*}
D_y J(y,u,v) h = & \int_{\Omega} (\phi_y(x,y) - p_1(T)) h(T) \; dx \\
& + \int_{0}^{T} \int_{\Omega} (\frac{\partial p_1}{\partial t} + \Delta p_1 - d_y(x,t,y,v) p_1 + \varphi_y(x,t,y, v)) h \; dx \; dt  \\
& + \int_{0}^{T} \int_{\Gamma} (p_1 - p_2) \frac{\partial h}{\partial n} \; ds \; dt \\
& + \int_{0}^{T} \int_{\Gamma} (- \frac{\partial p_1}{\partial n} - b_y(x,t,y,u) p_2 + \psi_y(x,t,y, u))  h \; ds \; dt
\end{align*}
Now we make special choices of $h$, first $h \in C_0^\infty(Q)$, then with arbitrary $h(T)$, arbitrary $h|_\Sigma$  and finally with arbitrary $\frac{\partial h}{\partial n}$, and we set $p := p_1$, $p_2 = p$ on $\Sigma$. From this we obtain the following adjoint equation:
\begin{equation}
\begin{alignedat}{2}
-p_t - \Delta p + d_y(x,t,\bar{y}, \bar{v}) p &= \varphi_y(x,t,\bar{y}, \bar{v}) &&\text{ in } Q \\
\partial_n p + b_y(x,t,\bar{y}, \bar{u}) p &= \psi_y(x,t,\bar{y}, \bar{u}) &&\text{ in } \Sigma \\
p(x,T) &= \phi_y(x, \bar{y}(x,T)) &&\text{ in } \Omega
\label{eq:adjoint-pde}
\end{alignedat}
\end{equation}
This can also be found in [Tr\"oltzsch, p.225f].\\
Finally we also compute $D_u \mathcal{L}$ and $D_v \mathcal{L}$:
\begin{align*}
D_u \mathcal{L}(y,u,v, p) h & = D_u J(y,u,v) h - \int_{0}^{T} \int_{\Gamma} b_u(x,t,y,u) p h \; ds \; dt \\
& = \int_{0}^{T} \int_{\Gamma}  (\psi_u(x,t,y,u) -  b_u(x,t,y,u) p)  h \; ds \; dt
\end{align*}
\begin{align*}
D_v \mathcal{L}(y,u,v, p) h & = D_v J(y,u,v) h - \int_{0}^{T} \int_{\Omega} d_v(x,t,y,v) p h \; dx \; dt \\
& = \int_{0}^{T} \int_{\Omega}  (\varphi_v(x,t,y,v) -  d_v(x,t,y,v) p)  h \; dx \; dt
\end{align*}

\section{Weak form of the adjoint system}
To derive the weak form of the adjoint system \eqref{eq:adjoint-pde} we use backward Euler and replace $p_t$ by $\frac{p_{k+1} - p_k}{h}$. We multiply with the test function $v$ and integrate:
\begin{align*}
- \int_{\Omega} \frac{p_{k+1} - p_k}{h} v \; dx - \int_{\Omega} \Delta p_{k+1} v \; dx + \int_{\Omega} d_y(x,t_{k}, y_k, v_k) p_{k+1} v \; dx = \int_{\Omega} \varphi_y(x,t,y_k, v_k) v \; dx
\end{align*}
Using partial integration on the second integral and inserting the boundary condition yields 
\begin{align*}
- \int_{\Omega} \Delta p_{k+1} v \; dx & = - \int_{\Gamma} \frac{\partial p_{k+1}}{\partial n} v ds + \int_{\Omega} \nabla p_{k+1} \nabla v \; dx \\
& = \int_{\Gamma} b_y(x,t_k, y_k, u_k) p_{k+1} v \; ds - \int_{\Gamma} \psi_y(x,t_k, y_k, u_k) v \; ds + \int_{\Omega} \nabla p_{k+1} \nabla v \; dx
\end{align*}
We insert this in the first equation again and order by integration domain
\begin{align*}
\int_{\Omega} (-\frac{p_{k+1} - p_k}{h} + d_y(x,t_{k}, y_k, v_k) p_{k+1} - \varphi_y(x,t,y_k, v_k)) v + \nabla p_{k+1} \nabla v dx & \\
+ \int_{\Gamma} (b_y(x,t_k, y_k, u_k) p_{k+1} - \psi_y(x,t_k, y_k, u_k) ) v \; ds  & = 0
\end{align*}

\subsection{Adjoint equation for our case}
In our case the equation simplifies to:
\begin{align*}
\int_{\Omega} (-\frac{p_{k+1} - p_k}{h} - (y_k - y_{\Omega,k})) v + \nabla p_{k+1} \nabla v \; dx 
+ \int_{\Gamma} \gamma p_{k+1} v \; ds  = 0
\end{align*}

Additionally, we have the initial (or rather terminal) condition:
\begin{alignat*}{2}
p_{N+1} &= &\phi_y(x, y_{N+1}) = y_{N+1} - y_{\Omega, N+1}
\end{alignat*}

\section{Solution by Projected Gradient Method}
Let $N \in \mathbb{N}$ be the MPC horizon and let $u^n := (u_0^n, u_1^n, \hdots, u_{N-1}^n)$, $v^n := (v_0^n, v_1^n, \hdots, v_{N-1}^n)$ be the iterates of the optimization algorithm. \\ 
The gradient of the reduced cost functional $f(v,u) = J(y(v,u), v, u)$ is given by
\begin{align*}
f'(v^n, u^n)(v,u) = & \int \int_Q (\varphi_v(x,t, y^n, v^n) - d_v(x,t,y,v) p^n) \; v \; dx \; dt \\
& + \int \int_{\Sigma} (\psi_u(x,t,y^n, u^n) - b_u(x,t,y,u) p^n) \; u \; ds \; dt
\end{align*}
This can also be found in [Tr\"oltzsch, p. 243f], for the special case of $d(x,t,y,v) = v$, $b(x,t,y,u) = u$.

Solution algorithm:\\
\begin{enumerate}
\item Solve forward system for given $(u^n, v^n)$ $\rightsquigarrow y_n$ \\
\item Solve adjoint system $\rightsquigarrow p_n$ \\
\item Descent directions
\begin{align*}
&h_n := -(\varphi_v(\cdot, y_n, v_n) - d_v(\cdot,y_n,v_n) p_n) \\ 
&r_n := -(\psi_u(\cdot, y_n |_{\Sigma}, u_n) - b_u(\cdot,y_n,u_n) p_n |_{\Sigma})
\end{align*}
\item Compute step size $\rightsquigarrow s_n$
(e.g. use $\min_{s > 0} f(\mathbb{P}_V(v_n+s h_n), \mathbb{P}_U(u_n + s r_n)).$
\item New iterates:
\begin{equation}
(v_{n+1}, u_{n+1}) := (\mathbb{P}_V(v_n + s_n h_n), \mathbb{P}_U(u_n + s_n r_n))
\end{equation}
\end{enumerate}

\section{Solver Options/Preconditioning}
\newpage
%----------------------------------------------------------------------------------------
%	BIBLIOGRAPHY
%----------------------------------------------------------------------------------------

\renewcommand{\refname}{Reference} % Change the default bibliography title

\bibliography{../bibtex} % Input your bibliography file
\bibliographystyle{plain}


%----------------------------------------------------------------------------------------

\end{document}
